\documentclass{article}

\title{Software Workshop 1 - Lecture 2}
\author{Ossama Edbali}

\begin{document}

	\maketitle
	
	\section{Higher-order functions}
	Higher-order functions are functions that do at least one of the following actions:
	\begin{itemize}
		\item takes one or more functions as a parameter
		\item outputs a function
	\end{itemize}
	
	The most notable examples are: map, fold and filter.
	The implementation in Java of higher functions using lists
	requires the following classes/interfaces:
	\begin{description}
		\item[List.java] Defines the interface for Nil and Cons
		\item[Nil.java] Defines the empty list (base case)
		\item[Cons.java] Defines the non-empty list (recursive case)
		\item[Action.java] Defines a procedure/action (i.e. does not return any value)
		\item[Function.java] Defines a 1-1 function (1 input, 1 output)
		\item[Function2.java] Defines a 2-1 function (2 inputs, 1 output)
		\item[Predicate.java] Defines a function that takes a single argument and returns a
		boolean
	\end{description}	
	
	Check src folder for the implementation.	
	
\end{document}