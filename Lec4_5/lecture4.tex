\documentclass{article}

\title{Software Workshop 1 - Lecture 4 \& 5}
\author{Ossama Edbali}

\begin{document}
	
	\maketitle
	
	\section{Fold}	
	Fold is an expressive way for applying a function in a cumulative way.
	The fold function replaces \verb|Cons|	with \verb|f| and \verb|Nil| with \verb|b| (which is the
	initial value). The polymorphic type of the function fold is: \verb|f: 'a -> 'b -> 'b|.
	
	\section{Binary trees}	
	Binary trees can be created using an interface and two implementations:
	\begin{itemize}
		\item Interface file: \verb|Tree|
		\item Base case implementation: \verb|EmptyTree<E>|
		\item Inductive case implementation: \verb|MakeTree<E>|
	\end{itemize}
	
	Binary trees have a root, right and left subtrees. Every node has at most 2 children where nodes
	with zero children are called leaves.	
	
	The selectors are:
	\begin{itemize}
		\item \verb|root| for getting the root of a BT
		\item \verb|left| for getting the left subtree.
		\item \verb|right| for getting the right subtree.
	\end{itemize}
	
	One might use the Maybe type in order to deal with these selectors (in case we have an empty tree we
	would return a \verb|Nothing| object).
	
	\section{Binary Search  Trees}	
	BSTs are a special case of binary trees where the values of the nodes in the left subtree must be smaller
	than the root (viceversa for the right subtree).
	
	The class/interface layout is:
	\begin{itemize}
		\item Interface file: \verb|Bst|
		\item Base case implementation: \verb|Empty<E>|
		\item Inductive case implementation: \verb|Fork<E extends Comparable<E>>| 
	\end{itemize}
	
	\section{Notes on generics}	
	Generics are a way in Java to implement polymorphic methods and classes.
	\begin{verbatim}
		public class List<E> {
		    ...
		}
	\end{verbatim}
	
	In the class List one can use the type \verb|E| in the methods declaration.
	But if we want to use a different type than \verb|E| in a method declaration we must use this:
	\begin{verbatim}
		...
		public <B> fold(Function<E, B> f, B init) {
		    ...
		}
	\end{verbatim}
\end{document}